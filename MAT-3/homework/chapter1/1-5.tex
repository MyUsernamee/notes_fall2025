
\documentclass[fleqn]{article}
\usepackage{amsmath}
\usepackage{shortex}
\title{1-5}
\begin{document}

\maketitle
\pagebreak
 Determine whether the given matrix is elementary.

\subparagraph{1}

\[
    \begin{array}{c c}(a) \begin{bmatrix} 1 & 0 \\ -5 & 1 \end{bmatrix} & (b) \begin{bmatrix} -5 & 1 \\ 1 & 0 \end{bmatrix} 
    \\ (c) \begin{bmatrix} 1 & 1 & 0 \\ 0 & 0 & 1 \\ 0 & 0 & 0 \end{bmatrix} & (d) \begin{bmatrix} 2 & 0 & 0 & 2 \\ 0 & 1 & 0 & 0 \\ 0 & 0 & 1 & 0 \\ 0 & 0 & 0 & 1 \end{bmatrix} \end{array}
\]
\vfill

 In Exercises 5–6 an elementary matrix E and a matrix A are given. Identify the row operation corresponding to E and verify that the product EA results from applying the row operation to A.

\subparagraph{5 a)}

\[
E = \begin{bmatrix} 0 & 1 \\ 1 & 0 \end{bmatrix}, \quad A = \begin{bmatrix} -1 & -2 & 5 & -1 \\ 3 & -6 & -6 & -6 \end{bmatrix}
\]
\vfill


\pagebreak


\subparagraph{b)}

\[
E = \begin{bmatrix} 1 & 0 & 0 \\ 0 & 1 & 0 \\ 0 & -3 & 1 \end{bmatrix}, \quad A = \begin{bmatrix} 2 & -1 & 0 & -4 & -4 \\ 1 & -3 & -1 & 5 & 3 \\ 2 & 0 & 1 & 3 & -1 \end{bmatrix}
\]
\vfill

 "Determine if $ad-dc\ne0$  then if it is  find the inversion of the matrix using the standard 2x2 formula. Afterwhich  use the inversion algorithm to find $A^{-1}$"

\subparagraph{9 b)}

\[
A = \begin{bmatrix} 2 & -4 \\ -4 & 8 \end{bmatrix}
\]
\vfill


\pagebreak
 Use the inversion algorithm to find the inverse of the matrix (if it exists)

\subparagraph{11 a)}

\[
 \begin{bmatrix} 1 & 2 & 3 \\ 2 & 5 & 3 \\ 1 & 0 & 8 \end{bmatrix}
\]
\vfill



\subparagraph{15}

\[
\begin{bmatrix} 2 & 6 & 6 \\ 2 & 7 & 6 \\ 2 & 7 & 7 \end{bmatrix}
\]
\vfill


\pagebreak


\subparagraph{17}

\[
\begin{bmatrix} 2 & -4 & 0 & 0 \\ 1 & 2 & 12 & 0 \\ 0 & 0 & 2 & 0 \\ 0 & -1 & -4 & -5 \end{bmatrix}
\]
\vfill

 Find the inverse of the matricies where $k_{1 \to 4}$ are all nonzero.

\subparagraph{19 b)}

\[
\begin{bmatrix} k & 1 & 0 & 0 \\ 0 & 1 & 0 & 0 \\ 0 & 0 & k & 1 \\ 0 & 0 & 0 & 1 \end{bmatrix}
\]
\vfill


\pagebreak
 Express the matrix and its inverse as products of elementary matricies.

\subparagraph{23}

\[
\begin{bmatrix} -3 & 1 \\ 2 & 2 \end{bmatrix}
\]
\vfill


\end{document}
                
