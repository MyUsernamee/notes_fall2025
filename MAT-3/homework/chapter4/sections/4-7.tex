\documentclass{article}
\usepackage{custom}
\usepackage{shortex}
\title{ 4-7 }

\begin{document}

\maketitle
\pagebreak

In Exercises 3- 4, determine whether $\mathbf{b}$ is in the column space of $A$ , and if so, express $\mathbf{b}$ as a linear combination of the column vectors of $A$
\subsubsection*{ 3) a) }

\[\quad A = \begin{bmatrix} 1 & 1 & 2 \\ 1 & 0 & 1 \\ 2 & 1 & 3 \end{bmatrix}; \quad b = \begin{bmatrix} -1 \\ 0 \\ 2 \end{bmatrix}\]
\vfill


\subsubsection*{ b) }

\[\quad A = \begin{bmatrix} 1 & -1 & 1 \\ 9 & 3 & 1 \\ 1 & 1 & 1 \end{bmatrix}; \quad b = \begin{bmatrix} 5 \\ 1 \\ -1 \end{bmatrix}\]
\vfill

\pagebreak
In Exercises 7–8, find the vector form of the general solution of the linear system $A\mathbf{x} = \mathbf{b}$, and then use that result to find the vector form of the general solution of $A\mathbf{x} = \mathbf{0}$.
\subsubsection*{ 7) b) }

\(
\begin{array}{rrrr}
    x_1+&x_2+&2x_3=&5 \\
    x_1+& +&x_3=&-2 \\
    2x_1+&x_2+&3x_3=& 3
\end{array}
\)

\vfill

In Exercises 9- 10, find bases for the null space and row space of A.
\subsubsection*{ 9 a) }

\(A = \begin{bmatrix} 1 & -1 & 3 \\ 5 & -4 & -4 \\ 7 & -6 & 2 \end{bmatrix}\)
\vfill

\pagebreak

\subsubsection*{ 10 a) }

\(A = \begin{bmatrix} 1 & 4 & 5 & 2 \\ 2 & 1 & 3 & 0 \\ -1 & 3 & 2 & 2 \end{bmatrix}\)
\vfill

In Exercises 11- 12, a matrix in row echelon form is given. By inspection, find a basis for the row space and for the column space of that matrix.
\subsubsection*{ 11 b) }

\[\begin{bmatrix}
1 & -3 & 0 & 0 \\
0 & 1 & 0 & 0 \\
0 & 0 & 0 & 0 \\
0 & 0 & 0 & 0
\end{bmatrix} \]
\vfill

\pagebreak

\subsubsection*{ 12 a) }

\[\begin{bmatrix} 1 & 2 & 4 & 5 \\ 0 & 1 & -3 & 0 \\ 0 & 0 & 1 & -3 \\ 0 & 0 & 0 & 1 \\ 0 & 0 & 0 & 0 \end{bmatrix}\]
\vfill


\subsubsection*{ 13) a) Use the methods of Examples 6 and 7 to find bases for the row space and column space of the matrix }

\[A = \begin{bmatrix} 1 & -2 & 5 & 0 & 3 \\ -2 & 5 & -7 & 0 & -6 \\ -1 & 3 & -2 & 1 & -3 \\ -3 & 8 & -9 & 1 & -9 \end{bmatrix}\]
\vfill

\pagebreak

\subsubsection*{ b) Use the method of Example 9 to find a basis for the row space of $A$ that consists entirely of row vectors of $A$. }


\vfill

In Exericses 16–17, find a subset of the given vectors that forms a basis for the space spanned by those vectors, and then express each vector that is not in the basis as a linear combination of the basis vectors.
\subsubsection*{ 17) $v_{1}=(1,-1,5,2),$ $v_{2}=(-2,3,1,0),$ $v_{3}=(4,-5,9,4),$ $v_{4}=(0,4,2,-3),$ $v_{5}=(-7,18,2,-8)$ }


\vfill

\pagebreak
In each part, let $A = \begin{bmatrix} 1 & 2 & 0 \\ 1 & -1 & 4 \end{bmatrix}$. For the given vector $\mathbf{b}$, find the general form of all vectors $\mathbf{x}$ in $R^3$ for which $T_A(\mathbf{x}) = \mathbf{b}$ if such vectors exist.
\subsubsection*{ 21 a) $\mathbf{b} = (0, 0)$ }


\vfill


\subsubsection*{ c) $\mathbf{b} = (-1, 1)$ }


\vfill

\pagebreak

\subsubsection*{ 24 a) Find a 3 × 3 matrix whose null space is a point. }


\vfill


\subsubsection*{ b) Find a 3 × 3 matrix whose null space is a line. }


\vfill

\pagebreak

\end{document}
