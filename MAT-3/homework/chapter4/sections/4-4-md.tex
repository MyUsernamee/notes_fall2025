\documentclass{article}
\usepackage{custom}
\usepackage{shortex}
\title{}

\begin{document}

\maketitle
\pagebreak

4-4

\section{\texorpdfstring{1. Use the method of Example 3 to show that the
following set of vectors forms a basis for
\(R^2\).}{1. Use the method of Example 3 to show that the following set of vectors forms a basis for R\^{}2.}}\label{use-the-method-of-example-3-to-show-that-the-following-set-of-vectors-forms-a-basis-for-r2.}

{[} \sbra{ (2, 1),\ (3, 0) } {]}

\section{\texorpdfstring{3. Show that the following polynomials form a
basis for
\(P_2\).}{3. Show that the following polynomials form a basis for P\_2.}}\label{show-that-the-following-polynomials-form-a-basis-for-p_2.}

{[} x\^{}2 + 1, \quad x\^{}2 - 1, \quad 2x - 1 {]}

\section{\texorpdfstring{5. Show that the following matrices form a
basis for
\(M_{22}\).}{5. Show that the following matrices form a basis for M\_\{22\}.}}\label{show-that-the-following-matrices-form-a-basis-for-m_22.}

{[}

\begin{bmatrix} 3 & 6 \\ 3 & -6 \end{bmatrix}

, \quad 

\begin{bmatrix} 0 & -1 \\ -1 & 0 \end{bmatrix}

, \quad 

\begin{bmatrix} 0 & -8 \\ -12 & -4 \end{bmatrix}

, \quad 

\begin{bmatrix} 1 & 0 \\ -1 & 2 \end{bmatrix}

{]}

\section{\texorpdfstring{7. a) In each part, show that the set of
vectors is not a basis for
\(R^3\).}{7. a) In each part, show that the set of vectors is not a basis for R\^{}3.}}\label{a-in-each-part-show-that-the-set-of-vectors-is-not-a-basis-for-r3.}

{[}\sbra{(2, -3, 1), (4, 1, 1), (0, -7, 1)}{]}

\section{\texorpdfstring{9. Show that the following matrices do not form
a basis for
\(M_{22}\).}{9. Show that the following matrices do not form a basis for M\_\{22\}.}}\label{show-that-the-following-matrices-do-not-form-a-basis-for-m_22.}

{[}

\begin{bmatrix} 1 & 0 \\ 1 & 1 \end{bmatrix}

, \quad 

\begin{bmatrix} 2 & -2 \\ 3 & 2 \end{bmatrix}

, \quad 

\begin{bmatrix} 1 & -1 \\ 1 & 0 \end{bmatrix}

, \quad 

\begin{bmatrix} 0 & -1 \\ 1 & 1 \end{bmatrix}

{]}

\section{\texorpdfstring{11. a) Find the coordinate vector of
\(\mathbf{w}\) relative to the basis
\(S = \sbra{\mathbf{u}_1, \mathbf{u}_2}\) for
\(R^2\).}{11. a) Find the coordinate vector of \textbackslash mathbf\{w\} relative to the basis S = \textbackslash sbra\{\textbackslash mathbf\{u\}\_1, \textbackslash mathbf\{u\}\_2\} for R\^{}2.}}\label{a-find-the-coordinate-vector-of-mathbfw-relative-to-the-basis-s-sbramathbfu_1-mathbfu_2-for-r2.}

{[}\mathbf{u}\_1 = (2, -4),~\mathbf{u}\_2 = (3, 8); \quad \mathbf{w} =
(1, 1){]}

\section{\texorpdfstring{13. a) Find the coordinate vector of
\(\mathbf{v}\) relative to the basis
\(S = \sbra{\mathbf{v}_1, \mathbf{v}_2, \mathbf{v}_4}\) for
\(R^3\).}{13. a) Find the coordinate vector of \textbackslash mathbf\{v\} relative to the basis S = \textbackslash sbra\{\textbackslash mathbf\{v\}\_1, \textbackslash mathbf\{v\}\_2, \textbackslash mathbf\{v\}\_4\} for R\^{}3.}}\label{a-find-the-coordinate-vector-of-mathbfv-relative-to-the-basis-s-sbramathbfv_1-mathbfv_2-mathbfv_4-for-r3.}

{[}\mathbf{v} = (2, -1, 3); \quad \mathbf{v}\_1 = (1, 0,
0),~\mathbf{v}\_2 = (2, 2, 0){]}

\section{\texorpdfstring{15. In Exercises 15- 16, first show that the
set \(S = \{A_1, A_2, A_3, A_4\}\) is a basis for \(M_{22}\), then
express \(A\) as a linear combination of the vectors in \(S\), and then
find the coordinate vector of \(A\) relative to
\(S\).}{15. In Exercises 15- 16, first show that the set S = \textbackslash\{A\_1, A\_2, A\_3, A\_4\textbackslash\} is a basis for M\_\{22\}, then express A as a linear combination of the vectors in S, and then find the coordinate vector of A relative to S.}}\label{in-exercises-15--16-first-show-that-the-set-s-a_1-a_2-a_3-a_4-is-a-basis-for-m_22-then-express-a-as-a-linear-combination-of-the-vectors-in-s-and-then-find-the-coordinate-vector-of-a-relative-to-s.}

{[}\quad A\_1 =

\begin{bmatrix} 1 & 1 \\ 1 & 1 \end{bmatrix}

, \quad A\_2 =

\begin{bmatrix} 0 & 1 \\ 1 & 1 \end{bmatrix}

, \quad A\_3 =

\begin{bmatrix} 0 & 0 \\ 1 & 1 \end{bmatrix}

,{]} {[}A\_4 =

\begin{bmatrix} 0 & 0 \\ 0 & 1 \end{bmatrix}

; \quad A =

\begin{bmatrix} 1 & 0 \\ 1 & 0 \end{bmatrix}

{]}

\section{\texorpdfstring{17. In Exercises 17- 18, first show that the
set \(S = \{p_1, p_2, p_3\}\) is a basis for \(P_2\), then express \(p\)
as a linear combination of the vectors in \(S\), and then find the
coordinate vector of p relative to
\(S\).}{17. In Exercises 17- 18, first show that the set S = \textbackslash\{p\_1, p\_2, p\_3\textbackslash\} is a basis for P\_2, then express p as a linear combination of the vectors in S, and then find the coordinate vector of p relative to S.}}\label{in-exercises-17--18-first-show-that-the-set-s-p_1-p_2-p_3-is-a-basis-for-p_2-then-express-p-as-a-linear-combination-of-the-vectors-in-s-and-then-find-the-coordinate-vector-of-p-relative-to-s.}

{[}p\_1 = 1 + x + x\^{}2, \quad p\_2 = x + x\^{}2, \quad p\_3 = x\^{}2;
{]} {[}p = 7 - x + 2x\^{}2{]}

\end{document}
