\documentclass{article}
\usepackage{custom}
\usepackage{shortex}
\usepackage{tikz}
\usepackage{tkz-euclide}

\usetikzlibrary{arrows}
\title{Chapter 6}

\setcounter{section}{6}

\begin{document}

\maketitle
\pagebreak

\subsection{}
\subsection{}
\subsection{Orthonormal bases.}

\bdefn

\textbf{Orthonormal} - This is when two vectors say, $\vec{u_{1}}$ and $\vec{u_{2}}$ are orthogonal (perpendicular), and both are unit vectors (of length 1). They are called \textbf{orthonormal}

\edefn

To convert two vectors to orthonormal, $\vec{u_{1}};\; \vec{u_{2}}$, which we are using as basis for some vector space $S$, we first project them to make them orthogonal, then we normalize both vectors, which we can then use as a coordinate system.

\bcent
Basis $\to$ Orthogonal $\to$ Orthonormal $\to$ Coordinate System.
\ecent

\textbf{From this point forward, orthogonal means orthonormal.}

When we want to perform a dot product we are used to the normal notation, but this is limited to Euclidian geometry. As a result we have some new notation.

\(
    \begin{array}{c|c}
        \text{Euclidian Notation} & \text{General Notation} \\
        \hfill \\
        \vec{u} \cdot \vec{v} & \lt< \vec{u}, \vec{v} \rt>
    \end{array}
\)


\subsubsection{Gran-Schmit Process}

This is process is used to form orthogonal vectors.

First we will start with an example in 2D.

\bexa

\[
    u_{1} && u_{2}
\]

\bcent
\begin{tikzpicture}
    \draw[thin, <->, dash pattern] (-4, 0) -- (4, 0) node[above] {$x$};
    \draw[thin, <->, dash pattern] (0, -4) -- (0, 4) node[above] {$y$};
    \draw[->] (0,0) -- (2, 1)
        node[above] {$u_{1}$};
    \draw[->] (0,0) -- (1, 2)
        node[above] {$u_{2}$};
    \draw[->, color=red] (0,0) -- (-1, 2)
        node[above] {$w_{2}$};
\end{tikzpicture}
\ecent

\eexa

\end{document}
