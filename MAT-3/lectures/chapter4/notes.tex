\documentclass{article}
\usepackage{custom}
\usepackage{shortex}
\title{Chapter 4}

\setcounter{section}{4}

\begin{document}

\maketitle
\pagebreak

\subsection{Real Vector Spaces}

\bdefn
We call $V$ a vector space if it follows these ten conditions.

\begin{enumerate}
    \item If $u$ and $v$ are object in V, then $u + v$ is in $V$ 
    \item $u + (v + w) = (u+v)+w$ 
    \item There is an object, $0$ in $V$, called a \textit{zero vector} for $V$, such that $0+u=u+0=u$ for all u in $V$.
    \item For each $u$ in $V$, there is an object $-u$ in $V$, called a \textit{negative} of $u$, such that $u + (-u)=(-u)+u=0$. 
    \item if $k$ is any scalar and $u$ is any object in $V$, then $ku$ is in $V$. 
    \item $k(u+v)=ku+kv$ 
    \item $(k + m)u = ku + mu$
    \item $k(mu)=(km)(u)$ 
    \item $1u = u$
\end{enumerate}


\brmku
Something to notice here is that this doesn't state anything about vector operations at, all. This truly does mean that any set with those 10 properties is a vector space. A great example of this is the set of real numbers, $\sdR$.
\ermku

To show a space is a vector space:

\begin{enumerate}
    \item Identify the set $V$ of objects that will become vectors. 
    \item Identify the addition and scalar mutliplication operations on $V$. 
    \item Verify Axioms 1 and 6; that is, adding two vector in $V$ produces a vector in $V$, and multipliying a vector in $V$ by a scalar also produces a vector in $V$. Axiom 1 is called \textit{closure under addition}, and Axiom 6 is called \textit{closure under scalar mutliplication}. 
    \item Confirm that Axioms 2, 3, 4, 5, 6, 7, 8, 9, and 10 hold.
\end{enumerate}

\edefn

\bthm
Let $V$ be a vector space, $u$ a vector in $V$, and $k$ a scalar; then:
\begin{enumerate}
    \item $0u=0$ 
    \item $k0=0$ 
    \item $(-1)u=-u$ 
    \item If $ku=0$, then $k=0$ or $u=0$.
\end{enumerate}
\ethm

\pagebreak

\subsection{Subspaces}

\bdefn
A subset $W$ of a vector space $V$ is called a $\textit{subspace}$ of $V$ if $W$ is itself a vector space under the addition and scalar multiplication defined on $V$.

Written as a set, we are essentially saying,

\[
    \cbra{W \mid W \subset V}
\]


\edefn

\bthm
If $W$ is a set of one or more vectors in a vector space $V$, then $W$ is a subspace of $V$ if and only if the following conditions are satisfied.

(a) If $u$ and $v$ are vectors in W, then $u$ + $v$ is in $W$.

(b) If k is a scalar and $u$ is a vector in W, then ku is in $W$.
\ethm



\end{document}
