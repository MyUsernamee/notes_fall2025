\documentclass{article}
\usepackage{custom}
\usepackage{shortex}
\title{Section 4.3 Linear Independence.}

\begin{document}

\maketitle
\pagebreak

Linear independence is when two elements do not change in relation to the other.
We can analyze this geometrically.
Let's imagine we have:

\[
    \cbra{u_{1}, u_{2}} \in \sdR^{2}
\]

\bdefn
\[
    S = \cbra{u_{1}, u_{2}, \dots, u_{m}}
\]

If $V_{f} = hV_{f}$, thwn these vectors are L.D. (Linearly dependent) to each other, otherwise they are L.I. (Linearly Independent)
\edefn
\hfill

\bthm
To decide vectors are LI, or LD. \\
\bcent
$S = {\vec{V_{1}}, \vec{V_{2}},\dots, \vec{V_{m}} }$ in $\skR^{n}$, C-scalar.\\
\ecent
\begin{enumerate}
    \item For $C_{1}V_{1} + C_{2}V_{2} + \dots + C_{m}V_{m} = 0$. \\
        \bcent
        \begin{tabular}{lc}
            If $C_{i} \ne 0$ &  $\to$ L.D.  \\
            If $C_{1} = C_{2} = \dots = C_{m} = 0$ & $\to$ L.I.  \\
        \end{tabular}
        \ecent
    \item $V_{i}=CV_{j}$ \quad LD (definition)
    \item 
        \[
            \vec{0}\in S  && L.D. \\
            C\vec{0} = 0 \\
            C\in \skR  \\
        \]
    \item $S=\cbra{\vec{V}}$ and $\vec{V}$ is non zero. L.I. \\
        \bcent
        $C\vec{V}=0$ \\
        $C = 0$ in $\vec{V}\ne0$ L.I. 
        \ecent
    \item \hfill 
        \bcent
        \begin{tabular}{lll}
            If $m > n$, & L.D. e.g. $\cbra{\vec{V_{1}}, \vec{V_{2}}, \vec{V_{3}}}$ & $\skR^{2}$ \\
            If $m < n$, & $\cbra{\vec{V_{1}}, \vec{V_{2}}}$ & $\skR^{2}$  No conclusion, use other methods. \\
            If $m=n$,& use determinent. &
        \end{tabular}
        \ecent
    \item Wronskia's method. \\
        \[
            & f, g &  \\
            w = & \begin{vmatrix} 
            f & g \\ f' & g'
        \end{vmatrix} && =0 \qquad \text{L.D} \\
                && && \ne 0 \qquad \text{L.I}
        \] \\
    Note: To span vectors in the same set, use LD vectors in the set. To span vectors in a space, use LI vectors in the set.
\end{enumerate}
\ethm

\subsection*{Examples}

\subsubsection*{1 a)}

\[
    u_{1} &= (-1, 2, 4) \\
    u_{2} &= (5, -10, -20) \\
    u_{2} &= 5u, \qquad \text{LD}
\]

\subsubsection*{2 b)}

\bcent
4 vectors for $\skR^{3}$ LD
\ecent

\subsubsection*{4 a)}

Determine whether LD or LI in $\skP_{2}$

\[
    2 - x + 4x^{2}, \quad 3 + 6x + 2x^{2}, \quad 2 + 10x - 4x^{2} && \text{Use determinent} \\
    \begin{vmatrix} 
    c & x & x^{2} \\ 2 & 3 & 2 \\ -1 & 6 & 10 \\ 4 & 2 & -4
    \end{vmatrix} = 39 \ne 0, \quad \text{LI}
\]

\subsubsection*{5 b)}

\[
    \begin{bmatrix} 
    1 & 0 & 0 \\ 0 & 0 & 0
\end{bmatrix}, \quad \begin{bmatrix} 
0 & 0 & 1 \\ 0 & 0 & 0
\end{bmatrix} , \quad \begin{bmatrix} 
0 & 0 \\ 0 & 0 \\ 1 & 0
\end{bmatrix}, \quad M_{23} \\
\]
\[
    aV_{1} + bV_{2} + c &= 0 \\
    a &= 0 \\
    b &= 0 \\
    c &= 0 \quad \text{LI} \\
\]

\subsubsection*{20.}

Use Wronskian to show that the functions $f_{1}(x) = e^{x}$, $f_{2}(x)=xe^{x}$, and $f_{3}(x)=x^{2}e^{x}$ are linearly dependent.

\[
    w = \begin{vmatrix} 
    e^{x} & xe^{x} & x^{2}e^{x} \\ e^{x} & e^{x}+xe^{x} & 2xe^{x}+x^{2}e^{x} \\ e^{x} & 2e^{x}+xe^{x} & 2e^{x}+4xe^{x}+x^{2}e^{x}
    \end{vmatrix}
\]

\subsubsection*{26}

\[
    S = \cbra{v_{1}, v_{2}, v_{3}} &\in V \qquad LD \\
    v_{4} \in V, \text{ not in } S \\
\]

\[
    \begin{bmatrix} 
    2 & 3 & 322 \\ 3 & 32 & 3 \\ 2 & 1 & 3
    \end{bmatrix} \\
\]

\bthm
Something really nice here $X\in\sdR$
\ethm


\end{document}
