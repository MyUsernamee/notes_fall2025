\documentclass{article}
\usepackage{shortex}
\usepackage{gensymb}
\usepackage{custom}

\title{Chapter 3}
\author{Delano Leslie}

\setcounter{section}{3}

\begin{document}

\maketitle
\pagebreak

\subsection{Orthogonal Trajectories}

Essentially curves where the tangent lines are orthogonal (perpendicular).

\textbf{Consider:}

\[
    y &= cx^{3} && \to && c=\frac{y}{x^{3}}
\]

%TODO: Graph

Essentially there are an infinte number of solutions.

\[
    \der{y}{x} &= 3cx^{2} = 3(\frac{y}{x^{3}})x^{2} \\
    \text{Subtituting $c$ back in.}
    \der{y}{x} &= \frac{3y}{x}
\]

\subsubsection*{Example}

Show that the curves $y=x^{3}$ and $x^{2} + 3y^{2} = 4$ are orthogonal at their points of intersection.

\[
    x^{2}+3y^2 &= 4 \\
    \frac{1}{4}x^{2}+\frac{3}{4}y^2 &= 1 \\
    \frac{1}{2^{2}}x^{2}+\frac{\sqrt{3}^{2}}{2^{2}}y^2 &= 1 \\
\]

%TODO: Graph

\[
    y&=x^{3} & x^{2} + 3y^{2} &= 4 \\
             && x^{2} + 3(x^{3})^{2} &= 4 \\
             && x^{2} + 3x^{6} &= 4 \\
             && \text{Inspection}& \\
             && x=1 &\quad y=1 \\
             && x=-1 &\quad y=-1 \\
\]

\[
    \der{y}{x}&=3x^{2} && 2x + 6y \der{y}{x} = 0 \\
    \der{y(1, 1)}{x} &= 3(1)^{2} = 3 && 2(1) + 6(1) \der{y}{x} = 2/6\\
    \der{y(1, 1)}{x} &= 3(-1)^{2} = 3 && 2x + 6y \der{y}{x} = 0 \\
\]

\subsubsection*{Definition:}
When all the curves of one family $G(x, y, c) = 0$ intersect orthogonally, all the curves of another family $H(x, y, c)=0$, then the families are said to be orthogonal trajectories.

Our goal is given a DE find another family that satisfies the above conditions.

\subsubsection*{Example:}

\[
    y&=c_1x && x^{2}+ y^{2} &=c_2 \\
\]

To find orthogonal families to other families\dots

\[
    \der{y}{x} &= f(x,y) && \der{y}{x}&=\frac{-1}{f(x,y)}
\]

%TODO: Graph
\subsubsection{}

Find the O.T. for $y=\frac{c_1}{x}$.

\[
    \der{y}{x} &= -\frac{c_1}{x^{2}} \\
    c_1&=xy \\
    &= -\frac{xy}{x^{2}} \\
    &= -\frac{y}{x} \\
\]

For O.T.

\[
    \der{y}{x} &= \frac{x}{y} \\
    y \d y &= x \d x \\
    \frac{1}{2}y^{2} &= \frac{1}{2}x^{2}  + c_2 \\
    y^{2} &= x^{2} + c_2 \\
    y^{2} - x^{2} &= c_2 \\
\]

\subsubsection{}

\[
    y &= \frac{c_1x}{1+x} \\
    \der{y}{x} &= \frac{(1+x)c_1 - (c_1x)}{( 1+x )^{2}} \\
    &= \frac{c_1(1+x - x)}{( 1+x )^{2}} \\
    &= \frac{c_1(1+x - x)}{( 1+x )^{2}} \\
    c_1 &= \frac{y(1+x)}{x} \\
        &= \frac{\frac{y( 1+x )}{x}}{(1+x)^{2}} \\
        &= \frac{\frac{y}{x}}{(1+x)} \\
        &= \frac{y}{x(1+x)} \\
\]

\[
    \der{y}{x} &= -\frac{(1+x)x}{y} \\
    &= -\frac{x+x^{2}}{y} \\
    &= -\frac{x+x^{2}}{y} \\
    y \, \d y &= x + x^{2} + c_2 \\
    \frac{1}{2}y^2 &= -( \frac{1}{2}x^{2} + \frac{1}{3}x^{2} + c_2  )\\
    \frac{1}{2}y^2 &= -\frac{1}{2}x^{2} - \frac{1}{3}x^{2} - c_2  \\
    \frac{1}{2}y^2 + \frac{1}{2}x^{2} - \frac{1}{3}x^{2} &= c_2  \\
    3y^2 + 3x^{2} - 2x^{2} &= c_2  \\
\]

\subsubsection{}

\[
    y &= ln(\tan x + c_1) \\
    y' &= \frac{1}{\tan x + c_1} (\sec ^{2} x) \\
    &= \frac{\sec^{2} x}{\tan x + c_1} (\sec ^{2} x) \\
    e^{y} &= \tan x + c_1 \\
    e^{y} - \tan x = &= c_1 \\
    y' &= \frac{\sec^{2} x}{\tan x + e^{y} - \tan x} \\
    &= \frac{\sec^{2} x}{e^{y}} \\
\]

\[
    \der{y}{x} &= -\frac{e^{y}}{\sec ^{2} x} \\
               &= -(e^{y} \cos ^{2} x) \\
               &= -e^{y} \cos^{2} x \\
\]

\subsection{Applications of Linear Equations}

\subsubsection*{Growth and Decay}

We can of course find the growth and decay of anything, (bacteria, human population, money, radioactive substances, etc\dots). We call it growth or decay when the rate of change of that thing is dependent on how many there is of that thing now.

\[
    \der{A(t)}{t} &= kA(t) && A(t_0) &= A_0 && A(t)>0 \\
    \int \frac{A(t)}{A} \quad \d A &= \int k \d t
    ln|A(t)| &= kt + c_1 \\
    A(t) &= e^{kt + c_1} \\
         &= e^{kt}e^{c_1} \\
         &= Ce^{kt}\\
\]

\subsubsection*{Bacteria Growth}

A culture initially has $N_0$ bacteria at $t=1$ hours. The number of bacteria is $\frac{3}{2}N_0$. If the growth rate is proportional to the number of bacteria present. How long until the bacteria triples?

\[
    N&=N_0e^{kt} \\
    \frac{3}{2}N_0&=N_0e^{k(1)} \\
    \frac{3}{2}&=e^{k} \\
    \ln \frac{3}{2}&=|k| && k>0 \\
    k&=\ln \frac{3}{2} \\
    N(t) &= N_0e^{\ln \frac{3}{2}t} = N_0(e^{\ln \frac{3}{2}})^{t} \\
         &= N_0\frac{3}{2}^{t}  \\
\]

\[
    N(t) &= 3N_0 \\
    N_0\frac{3}{2}^{t} &= 3N_0 \\
    \frac{3}{2}^{t} &= 3 \\
    \frac{1}{2}^{t} &= 1 \\
\]

\subsubsection*{Half-Life $\lambda$}

\subsubsection*{Example}

The half-life of $U-238$ is $4.5x10^9$ years.
A breeder reactor converts $U-238$ into an isotope of plutonium 239. In 15 years it is determined that $0.043\%$ of the initial amount as disintegrated. Find the half-life.

\[
    m(t) &= m_0 e^{\lambda t} \\
    (\Delta m)m_0&= m_0 e^{\lambda t} \\ 
    (\Delta m)&= e^{\lambda t} \\ 
    e^{\Delta m}&= \lambda t \\ 
    \frac{ e^{\Delta m} }{t}&= \lambda \\ 
    \Delta m &= m_t - m_0
\]

\subsubsection*{Newton's Law of cooling}

The rate at which the temp. changes in a cooling body is proportional to the difference in the temp. of the temp of the body and the temp of the surrounding medium (air).

\[
    T &= T_m + Ce^{kt} 
\]

\subsubsection*{Bake a cake.}

$300\degree F$ when removed air temp is $70\degree F$. In 3 minutes, it cools off to $200\degree F$. How long until it cools off to $80\degree F$.

%TODO: THIS

\end{document}
