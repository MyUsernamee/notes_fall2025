
\documentclass[fleqn]{article}
\usepackage{amsmath}
\usepackage{shortex}
\title{3-2}
\begin{document}

\maketitle
\pagebreak


\subparagraph{1}

The population of a certain community is known to increase at a rate proportional to the number of people present at time t. If the population has doubled in 5 years, how long will it take to triple? to quadruple?

\vfill



\subparagraph{3}

The population of a town grows at a rate proportional to the population at time t. Its initial population of 500 increases by 15\% in 10 years. What will be the population in 30 years?

\vfill


\pagebreak


\subparagraph{4}

The population of bacteria in a culture grows at a rate proportional to the number of bacteria present at time t. After 3 hours it is observed that there are 400 bacteria present. After 10 hours there are 2000 bacteria present. After 10 hours there are 2000 bacteria present. After 10 hours there are 2000 bacteria present. after 10 hours there are 2000 bacteria present. that is the initial number of bacteria?

\vfill



\subparagraph{5}

The radioactive isotope of lead, Pb-209, decays at a rate proportional to the amount present at time t and has a half-life of 3.3 hours. If 1 gram of lead is present initially, how long will it take for 90\% of the lead to decay?

\vfill


\pagebreak


\subparagraph{6}

Initially there were 100 milligrams of a radioactive substance present. After 6 hours the mass had decreased by 3\%. If the rate of decay is proportional to the amount present at time t and has a half-life of 3.3 hours. If 1 gram of lead is present initially, how long will it take for 90\% of the lead to decay?

\vfill



\subparagraph{7}

Determine the half-life of the radioactive substance described in Problem 6.

\vfill


\pagebreak


\subparagraph{10}

When interest is compounded continuously, the amount of money S increases at a rate proportional to the amount present at time t: dS/dt = rS, where r is the annual rate of interest (see (26) of Section 1.2).

\vfill



\subparagraph{13}

A thermometer is removed from a room where the air temperature is 70°F to the outside where the temperature is 10°F. After $\frac{1}{2}$ minute the thermometer reads 50°F. What is the reading at t = 1 minute? How long will it take for the thermometer to reach 15°F?

\vfill


\pagebreak

\pagebreak
\subsection*{Answers:}


\subparagraph{1}

7.9 yr;\quad 10 yr 

\vfill



\subparagraph{3}

\[
760 
\]

\vfill



\subparagraph{4}

Not Provided

\vfill



\subparagraph{5}

\[
11 h 
\]

\vfill



\subparagraph{6}

Not Provided

\vfill



\subparagraph{7}

\[
136.5 h 
\]

\vfill



\subparagraph{10}

Not provided

\vfill



\subparagraph{13}

T(1) = 36.67°; approximately 3.06 min

\vfill


\end{document}
                
