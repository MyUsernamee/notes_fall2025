\documentclass{article}
\usepackage{custom}
\usepackage{shortex}
\title{Unit 4 In Class Assignment}

\begin{document}

\maketitle
\pagebreak

\begin{enumerate}
    \item The Sum rule is applicable when you have two sets of possible actions $A$ and $B$. Where $A \subset S$, $B \subset S$, but $A \cap B = 0$. The possible set of choices is $|A| + |B|$
    \item \begin{enumerate}
        \item $4(5) = 20$
	\item $4 + 5 = 9$ 
	\item $(3 + 2 + 1)(5) = 30$
    \end{enumerate}
    \item The product rule is applicable when you have two sets with the same set of restrictions as before, but you can perform one action from each set simultaneously.
    \item You would have $39^{3}$ possible combinations.
    \item The software must attempt $( 26 + 26 + 10 )^{7}$
    \item $(26 + 26 + 10 + 16)^{8}$
    \item In the worst case scenario $0.5(26 + 26 + 10 + 16)^{8}$ seconds, but it could also pick the right one on the first try, which would take $0.5$ seconds; On average, it would take $0.25(26 + 26 + 10 + 16)^{8}$ seconds.
    \item $(52^{8} - 52)(10^{8} - 10)$
\end{enumerate}

\end{document}
