
\documentclass{article}
\usepackage{custom}
\usepackage{shortex}
\title{Unit 4 In Class Assignment}

\begin{document}

\maketitle
\pagebreak

\begin{enumerate}
    \item The Sum rule is applicable when you have two sets of possible actions $A$ and $B$. Where $A \subset S$, $B \subset S$, but $A \cap B = 0$. The possible set of choices is $|A| + |B|$
    \item \begin{enumerate}
        \item $4(5) = 20$
	    \item $4 + 5 = 9$ 
	    \item $\binom{4}{2}(5) = 30$
    \end{enumerate}
    \item The product rule is applicable when you have two sets with the same set of restrictions as before, but you can perform one action from each set simultaneously.
    \item $41\cdot 40\cdot 39 = 63{,}960$
    \item \begin{enumerate}
        \item $62^{7}$
        \item $78^{8}$
        \item Worst case: $0.5(78^{8})$ seconds, Best case: $0.5$ seconds, Average case: $0.25(78^{8})$ seconds
        \item $62^{8} - 52^{8} - 10^{8}$
    \end{enumerate}
    \item \begin{enumerate}
        \item $30+40+50 = 120$
        \item $30 \cdot 40 \cdot 50 = 60{,}000$
        \item $\binom{50}{5} = 2{,}118{,}760$
        \item $\binom{120}{5} = 190{,}578{,}024$
    \end{enumerate}
    \item \begin{enumerate}
        \item $\lfloor \tfrac{987}{8} \rfloor - \lfloor \tfrac{99}{8} \rfloor = 111$
        \item $\lfloor \tfrac{987}{2} \rfloor - \lfloor \tfrac{99}{2} \rfloor = 444$
        \item $648$
    \end{enumerate}
    \item $\lceil 38/7 \rceil = 6$
    \item A permutation is an ordered arrangement of distinct objects. The number of permutations of $k$ elements chosen from $n$ is $P(n,k) = \dfrac{n!}{(n-k)!}$.
    \item \begin{enumerate}
        \item $P(5,3) = 60$
        \item $\binom{5}{4}\binom{2}{2}\binom{3}{3} = 5$
        \item $5! \cdot 2! = 240$
    \end{enumerate}
\end{enumerate}

\end{document}
end{document}
